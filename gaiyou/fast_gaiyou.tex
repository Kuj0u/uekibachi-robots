\documentclass[10pt]{jsarticle}
\usepackage{gaiyou}
\usepackage[dvipdfmx]{graphicx}
\usepackage[top=2cm, bottom=2cm, left=1.5cm, right=1.5cm]{geometry}

\begin{document}

\pagestyle{empty}

\setlength{\baselineskip}{13truept}

\title{植物自律化による自己育成システム}
\engtitle{Self-cultivation system by plant autonomization}
\supervisor{村松 聡}
\author{黒木 駿太}
\engsupervisor{Satoshi Muramatsu}
\engauthor{Shunta Kuroki}

\maketitle
\thispagestyle{empty}

\section{はじめに}
近年の植物は栽培方法が多様化し、屋内の植物工場までも作られるようになている。
室内では環境が年間を通して一定に調整することが容易で、生産のロスが少ないため、商売としても安定した運用が可能な点もメリットである。
しかし、工場の建設するにあたって莫大な資金と時間を必要とすることや、結局は人間が管理を行わければならない問題があります。
これらの問題点はロボット技術の発展で緩和はあるものの、根本的な解決にはならない。

そこで本研究では人間による管理を必要とせず、植物の自律的な行動で自己成長が行えるシステムを提案し、これを実現するために必要となるロボットシステムのハードウェアとソフトウェアを制作する。

\section{植物自律化}
本研究における植物自律化とは、ロボットを用いて植物を行動させることを指す。
元来、植物は環境に対して受動的なものであり、根を張った場所の良し悪しでその後の成長に多大な影響を及ぼすと言っても過言ではない。
一方、動物は自ら行動することによって置かれている環境から移動し、より有利な場所に自らを置くことができる。
環境の変化に対して柔軟に対応し、効率的な成長と種の存続の可能性が高まる。
そのような自律移動をロボット使い植物に行わせることによって、従来の植物では得ることができなかった移動手段と知能を表現すのが図(まだ)になります。

\section{システム}
本システムでは事前に作成した地図上で行動を行い、成長により有利な場所に移動するようプログラムされている。
地図情報は移動を行うごとに随時追加され、時間とともに情報の信頼性と予測精度が向上する。
本システムでは植物を搭載し走行する自律移動ロボット(図1的な後で)と、水分を供給する給水ステーション(zu2)にわかれている。

\subsection{自律移動ロボット}
移動を行うために対向2輪型ロボットを使用し、鉢植えを搭載できるサイズとなっている。
環境情報を取得するセンサ類を搭載することで自律移動の手助けを行っている。
ロボットの制御には小型コンピュータを中心としたものになっており、ネットワークを介することで他の機器とも連携が容易となっている。
\subsubsection{ハードウェア}
ここにハードを書く
今回使用するロボットは多くが手作りであり、筐体は2段になっている。
下層部はコンピュータや電装類を設置しており、上層部には植木鉢やセンサ類を設置するようになっている。
電装系の概略図を図に示す。(まだ書いてない。)
コンピュータは汎用性と省スペース性に優れたRaspberry Piを搭載。
アクチュエータとはドライバICを介して接続。
外界センサとして照度、温湿度、土壌をモニタリングし、ArduinoはADコンバータ代わりに使用している。
\subsubsection{ソフトウェア}
自律移動のためのソフトウェアの多くはRaspberry Pi上で動いており、センサで読み込んだ値を元にモータの制御を行っている。
センサ情報は2次元マップに関連付けて保存を行い、センサの数分3次元的に埋め込んでいる。それらの情報は時間で管理され最終的に4次元データで管理されます。(余裕があれば画像で保存的なことも)
自己位置推定にはロータリエンコーダを用いたオドメトリで計算を行い、細かい誤差が乗ってしまうものの無視することにしている。

\subsection{給水ステーション}
植物が自律的に成長するため水分の摂取は必須である。
水分を自発的に摂取するためにも、水を提供するロボットは必要なサービスである。
本ロボットは比較的コンパクトに作られており、様々な場所に設置が可能となっている。
また、ノズルの位置の調整を行うことで様々なシチュエーションに対応できます。
水はタンクに予め貯めておき、植物が必要となったタイミングで移動ロボットが接近し、バルブを開放し水を与える機構(図00)となっている。
ロボットの制御にはArduinoを使用し、サーボモータと距離センサの組み合わせで動作する。

\section{実験}
\subsection{実験方法}
本システムの評価方法として、限定空間内で自律移動を行った植物とそうでない植物の育成具合を比較する。
環境は冬場の室内$9m^2$とし、日の光を表現するためにスポットライトを時間ごとに動かし、温かい場所も作った実験環境は図00になる。
これらの環境で2週間運用する。
\subsection{実験結果及び考察}
実験結果は移動を行わない植物が一方的に枯れた。
(冬場だとそんな気がした。実験環境は後ほど考えます。)
自律移動を行う植物と定点固定の植物の比較画像は図00になる。
\section{おわりに}
ここは結言です
ここも後ほど書きます。
画像を入れるももう少しわかり易いのかと思いますが、日本語が崩壊していると思います。

\begin{thebibliography}{10}

%%参考文献
\bibitem{Paper01}
東海太郎, 東海二郎: XXXに基づくロボットのシステム制御の開発,
日本ロボット学会誌,{\bf xx}-xx, pp.xxx-xxx (20xx)

\end{thebibliography}

\end{document}
