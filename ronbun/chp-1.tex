\chapter{はじめに}
\section{背景}
%農業について語る。
生物が生存するために外部から栄養や水分を摂取することは必須である。
人類においても同様に栄養の摂取、つまり食事を摂取を日々行っている。。
これは図に示すマズローの自己実現理論の最底辺に定義されているように、知能が高度に発展した人類においても栄養の摂取は生命を維持するためには必須条件である。
\subsection{食糧事情}
%http://www.maff.go.jp/j/pr/annual/pdf/syoku_jijyou.pdf
%ほとんどこのpdfから画像とか数値とか資料ここから
食事に必要な作物は田畑で作られ、生産物は物流に乗り個人に供給されている。
この物流網はグローバル化に伴い国境を跨ぎ、海外からも供給されている。
近年では海外と比較し生産コストや田畑に適した土地が少ない問題により、多くの作物が海外より輸入されることが目立つ。
図に示すのは日本の食糧自給率の推移をグラフで表したもので、過去20年ほどは食糧自給率が40\%程を彷徨っている。
近年では39\%にまで低下しており、日本の食料事情は非常に不安定となっている。
食料を自国で生産できない分は必然的に海外より輸入することになる。
しかし

\section{従来研究}
%ロボット化、機械化、流行ってるね。
%でも、結局,植物は他者に依存してるね。
%それって、危ないでしょ。気候変動で種の滅亡とかあるでしょ。あかん。
\section{研究目的}
%自律的に行動して、他者に依存しない動物的な行動するればいいかも
%自分で光や水を求めて行動して、自分で生き抜く。雨乞いしない。
%それを実現できる物をつくる。
\section{論文構成}
%最後に書こう。
