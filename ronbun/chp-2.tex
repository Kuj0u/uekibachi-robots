\chapter{植物の自律化}
%そもそも植物は進化の経緯で今の状態。
%しかし、それではもったいない的な、植物を自律的に動かすといいかも、手足が生えて自分で動けると
植物自律化とは、能動的に行動できる植物を意味する。
元来、植物は大地に根を張り動かないことが特徴であるが、本研究ではそれと相反する動物の能動的な行動様式を取り入れ、他者に依存せず自律的に自己育成を行う。
具体的にはセンサにより自己状態を把握し、必要に応じてロボットで移動することでこれを実現する。
土壌の水分が減ると水分補給を行い、日差しが弱い場合にはより明るい場所を探すよう能動的に行動を行う。
それにより、動物のように自身の状態によって自由に行動し、他者に依存することのない生命活動を実現する。
\par また、本研究において動物は能動的に行動を行え、対して植物は環境に対して受動的なものとした括りで説明を行う。

\section{自律とは}
自律という言葉を辞書で調べると次のような意味合いを持つ。
\begin{quote}
\begin{enumerate}
    \item 他からの支配や助力を受けず,自分の行動を自分の立てた規律に従って正しく規制すること。 「学問の-性」

    \item {哲} 〔ドイツ Autonomie〕 カント倫理学の中心概念。自己の欲望や他者の命令に依存せず,自らの意志で客観的な道徳法則を立ててこれに従うこと。
\end{enumerate}
\end{quote}
本研究における自律とは、外界の情報をもとに自身の頭脳で思考し、その結果に基づき行動を行うことが自律的な行動と位置づける。
\par 余談になるが、自律と自立は厳密には異なった意味合いになる。
自立という言葉を辞書で調べると次のような意味合いを持つ。 
\begin{quote}
\begin{enumerate}
    \item 他の助けや支配なしに自分一人の力だけで物事を行うこと。ひとりだち。独立。 「親もとを離れて-する」

    \item 自ら帝王の位に立つこと。 「其後-して呉王となる/中華若木詩抄」 → 自律(補説欄) ・ 独立(補説欄)
\end{enumerate}
\end{quote}
つまり、図00のように自立は自律の部分集合であり、行動は行えるものの、自ら思考し答えに辿り着くような高度な知能は持ち合わせていない。


%言葉の問題だね。
%難しいね。
\section{動物的行動}
%動物っぽい行動でのメリットを述べる。
%植物も進化して高効率だけど、それに動物加えたら強いよねって
%自身の生命維持に必要なものを積極的に取りに行く、必要なら動くって素敵。
%そんなことを書く。
%どう書けばいいかわかんないけど。
人類を含めた動物はこの自律的な行動を獲得し、今日に至るまで進化しつづけ生存している。
動物が生命を維持するため、呼吸や排泄、栄養や水分の摂取が必要不可欠になる。
動物は種の存続を行うため、他者を捕食し自身の糧にし生命活動を行い、子孫を残した後にやがて死にゆく。
これは太古からの自然なサイクルであり、地球が誕生し陸上で恐竜が活動していた数億年以上前から行われている。
自然界のなかで自らの個体を維持するため、動物は様々な進化を遂げ種の存続を行っている。
\par 今日まで動物は進化を繰り返し、その過程において自律的行動を行う知性を獲得したとされる。
動物は過酷な環境に耐えうる手段を学習し、天敵となる捕食者からの逃れる術を体得した。
狩りにおいて仲間と連携を図ることで成功率を向上させる動物も存在する。
それら動物の特性をいくつか例をあげる。
\subsection{サーバルキャットの特性}
図00に示すサーバルキャットはサバンナ地方に生息するネコ科の動物である。
身体的な特徴は胴体と四肢が細く、素早く走行することと高く跳ねることができる。
体毛はサバンナの動物に多い黄褐色に黒斑点模様で周囲に同化しやすく、獲物や天敵から身を隠しやすい特徴がある。
活動範囲は広く、平原から山岳の標高2,000mを超える場所でも姿を表す。
基本的に夜行性であるが、朝夕の涼しい時間帯でも活動をしている。
\par 狩りは単独で行い、身軽さを武器に小型動物を仕留め、稀に低空を飛行している鳥類も捕食する。
単独行動は種に対し多様性をもたらし、様々な環境に対し耐性を身につけることが可能である。
\par しかし、時折金切り声を叫ぶこともあり、これは他の仲間とコミュニケーションを図っているとされている。
完全なスタンドアロンではなく、必要に応じて仲間と連絡を取ることで無駄を省いている(これは適当。ワカンネ)。

\subsection{ブチハイエナ特性}
図00に示すブチハイエナもサバンナ地方に生息する動物である。
外見的特徴よりイヌ科に似ているが、ネコ科に近い動物とされている。
また、狩りの獲物を横取りするイメージが先行しがちだが、実に6割以上は自らで獲得している。
むしろ、百獣の王ライオンがハイエナの獲物を横取りする方が多いと報告に上がっている。
\par ハイエナの狩りの特徴は仲間との協力プレイです。
知能の高い動物であり、10頭近いコミュニティーで常に行動を共にし、狩りの際にも仲間と連携を密に取り合うことで狩りを成功に導きます。
ハイエナの狩りの成功率は他の動物よりも高い水準にあり、これは種の存続に大きく貢献している。
同種族の他のコミュニティーとの喧嘩を避け、コミュニティー内も秩序が存在しひとつの社会として機能している。

\subsection{捕食-被食関係}
前節であげた2つの動物は捕食-被食関係に位置している。
天敵に狩られる立場でもあり、他を捕食する立場でもある。
これは常に身を危険に晒している状況にある。
つまり、安易な行動で天敵に捕食され、目的を前にして息絶えることとなる。
\par これらの脅威に対し動物は自律的な行動を持って対処している。
例えば複数の仲間と行動を共にすることによって、個が存続する確率を高めている。
また、発達した感覚器官から得た情報を元に、天敵が接近する前にこれを回避する。
そのために四肢が発達した動物も多い。
更には、過去の記憶を元に次の行動を予測し、自身の生存の確率を向上している。
この危険察知能力の要は卓越した感覚器官を器用に使いこなすことで、周囲の環境情報が取得可能となる。
\par 動物は自律的な行動を行うことで生存の確率が高まる。
適切な判断を行うためには周囲の環境情報が必要となり、適切な感覚器官が備わっている必要がある。
また、行動を実行に移すためには移動手段がが必要となり、環境に合わせた適切な四肢が動物には備わっている。

\section{育成システム}
%それをロボットで表現する。
%コンピュータで云々かんぬん。
本研究ではこれら動物的な行動、つまり自律性を植物に付与することで、従来の植物で問題となっている他者への依存度を低減させる。
しかし、現状の植物のみでは自律することは難しく、外的に機能を追加することでこれを克服する。
より具体的には、環境情報が知覚し処理しやすいよう外界センサを搭載しこれをコンピュータで処理することで、仮想的に感覚器官と頭脳を有しているように振る舞うことができる。
また、移動を行うために植物は移動ロボットに搭載し、これをコンピュータ制御することであたかも植物が自律性を有し、能動的な活動を行っているよう環境に対し適当な行動を行わせる。
これら自律的な育成を行うために必要となる、自己育成をシステムを本研究で作製し評価を行う。
\par また、本研究では育成を行うものであり、作付や収穫また繁殖は本研究では対象外とする。

