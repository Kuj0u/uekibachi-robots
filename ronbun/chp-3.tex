\chapter{システム}
こんな感じのシステムが必要になる。図とか?出しながら、
それには移動ロボットと水やりロボットが必要ね
それに、地図があってセンサも紐付ける必要あるね
\section{自律移動ロボット}
植木鉢を載せて動くロボット必要ね
こんな感じのロボットが必要ね。
\subsection{ハード}
全体的な筐体とか図を見せながら軽く触れる
電装系も概要のfig2を出す。
\subsubsection{駆動部}
対向2輪型を選んだ理由
要の駆動部、計算して出したんだよ
重くてもOKってのを見せる。
計算の結果このトルクとか回転数が必要になった
タイヤは多少の段差、室内限定ではコードを乗り越えられる的な感じで乗り切る
ぶっちゃけ時速は10km/h出るけど、仕方ない。適当にする。
そしたら12vバッテリーで駆動できていいじゃんってのをまとめる。
\subsubsection{モータドライバic}
このモータ動かすためにIC必要ね
モータ電流値をだす。
この値です。
ならこれ以上容量必要ね
DCならPWMで速度制御しやすいね、だからこれ
後述するけど、ソフトでもPWMやりやすいね
\subsubsection{エンコーダ}
正直パルス多い。200p/r
精度良くなら1000もあるけど、どうして200?
決めてなかった。ただ、手元にあった。それだけ。
んー。4低倍で800。更にギアかまして倍の1600。んー
モータとタイヤ直結だからそこそこのパルス数は必要
だけど、コンピュータ的に多すぎるのはダメだkら、、、このくらい?根拠ないけど。
適切なパルス数をどこかの文献から引っ張る?
適切に参考文献使うか。
それで解決しよう。ダメでもどうにか
\subsubsection{制御コンピュータ}
linuxが走るコンピュータ欲しかった。それはプラットフォームが云々、今後のROSとか詳しいことは後述で
さらに直接ピンが出てれば直接操作できていいね。的な。
それでRPI3が小型で低消費電力で安価だった。比較の必要はないと思う、これはみんなわかってる。(と思う。
\subsubsection{センサ}
環境認識にセンサの搭載は必須である
今回は動物のように状態を認知できることに着眼し、照度、水分、温湿度の3つのセンサを搭載した(これはもっと前の方で行ったほうがいいかも)
それぞれのセンサはアナログ値で出力される。これは後述するArduinoでAD変換され制御コンピュータに送られる。
\subsubsubsection{照度センサ}
照度センサにはCdSセルを利用し、照度の変化により抵抗値が変化するようになっている。
回路図は以下のようになっている。
固定抵抗の000Ωは実験の結果この値だと低すぎず高すぎず、できるだけ多くの変化を観測することが可能な値になっている。
(って、いう時どう書けばいいかな?)
また、このモジュールを四隅に3つ搭載している。これはセンサが植物の葉で日陰に入ってしまう事を考慮している。
移動することで照明の位置が常に変化し、1つのセンサでは対処しきれないため複数箇所に設置をしている。
\subsubsubsection{土壌水分センサ}
土壌水分の計測には図00のように土に2本の電極を刺し、抵抗値の変化を水分量の変化として捉えている。
土壌水分が多い場合は抵抗値が低くなり多くの電流が流れ、逆に水分量が少ない場合は抵抗値が高くなり電流が流れにくいようになっている。
この抵抗値は2本の電極の間隔が非常に重要なため、電極は次の図00のような音叉の形をした物を利用する。
間隔が固定されるため差し替え時による誤差は少なくなる。
この値をコンパレータLM393に通して出力をしている。

\subsubsubsection{温湿度センサ}
環境の温湿度を認知するためにセンサを搭載した
測定範囲は日本の平均的な(参考文献、気象庁)をもとに、次のようなものが求めれた。
なのでこのセンサなら
温湿度センサは市販品のDHT11を
使った。
設置場所は百葉箱を意識し、塩ビの中に入れた。(正直、コンピュータとかで温まっちゃうけどさ笑)
\subsubsubsection{Arduino}
これらセンサの値をAD変換しRaspberryPi3に送る必要がある。
まず、RPi3はアナログ入力は対応していない、なのでAD変換が必要
次に外部からの送り方は個別にGPIOでもいいが、センサ情報は同期取りたいので1つの機器でまとめて送る(なぜ?って聞かれたら微妙だけど)
まとめて送るためにはspiとかserialあるけど、今回は汎用的なSerial通信にした。
これらをまとめて搭載し、手軽にプログラミングできて、安価なのがArduinoでした。(他にもあるかもだけど)
(情報量多いし、資産としても悪くないかなって、稲垣先生も納得するかもって。わかんかいけど)
詳しいソフトは後述する。
\subsubsection{DC-DC コンバータ}
これらに電源を供給するものが必要になる
12Vのモータと同じ電源を利用する、サイズダウンのため
なのでコンバータが必要になる。
そこで、コンバータの必要スペックを計算する
5Vの各機器必要電流値を表で出す。
で、こんだけの電流が必要
なので、余裕のあるこれにした。
\subsubsection{バッテリー}
移動して運用するためにはバッテリーが必要となる。
上記のシステムをN時間稼働すると0000AH以上のバッテリーが必要となる。
計算式
なのでこれを選んだ

\subsection{ソフト}
細かい機能の紹介をすればいいのかな?
環境見ながらの自律移動に必要な機能に分けて掘り下げる感じ?
ここはちょっと端折る。頭が冴えている時に書く。骨組みを冴えてる時に。

\section{給水ステーションシステム}
存在意義の説明
ハードと
ソフトの説明


で、システムについては終わり.

