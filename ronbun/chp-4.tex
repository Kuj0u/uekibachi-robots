\chapter{機能・ソフト}
前章で用意されたハードウェアを用いて、システムは所定の動作を行う。
Fig.\ref{soft_zentai}にシステム概要図を示す。
2つのロボットは個別に動作をしており、通信などを特に行っていない。
\par 2つのロボットシステムを起動させると、基本的には待機状態になる。
自律移動ロボットが現状に不満を抱くと、事前に用意されているデータを基に思考し、最適な環境に移動するようなシステムになっている。
水分不足の場合には給水ロボットに近づき、給水ロボットはホースのバルブを開放し植物に水分を供給する。
そのような動作を繰り返し行うことで、自律移動ロボットに搭載された植物があたかも、自律的に行動しているように振る舞う。

\section{自律移動ロボット}
自律移動ロボットは初めに実験環境情報を取得する。
これは実験を行う環境内で移動を行い、座標に基づいたセンサ情報を取得し、それをLogデータとしてコンピュータ内に保存する。
生成されたlogデータを管理しやすいよう図00に示すグリット形式に変換する。
当たらに作られたデータを過去の記憶として、システムが動作しているロボットは現在の状態と過去の状態を比較し、より有利に働くよう行動を行う。


\subsection{環境情報取得}

\subsubsection{座標取得}

\subsubsection{センサ情報取得}

\subsubsection{センサ管理・GIS}

\section{状況判断}

\section{給水ロボット}
給水ロボットは終始待機状態になる。
待機状態ではホースのバルブを閉じた状態になるよう、サーボモータを固定した状態で待機している。
ロボットが接近した際にはバルブを開放方向へ回し、一定時間後に再びバルブを閉じた状態に戻し、一定時間は給水を受付ないようなっている。
\subsection{水分供給}

